\section* {2.1  Метод простой итерации и метод Ньютона для решения нелинейных уравнений}

\subsection{Постановка задачи}
Реализовать методы простой итерации и Ньютона решения нелинейных уравнений в виде программ, задавая в качестве входных данных точность вычислений. С использованием разработанного программного обеспечения найти положительный корень нелинейного уравнения (начальное приближение определить графически). Проанализировать зависимость погрешности вычислений от количества итераций. 

{\bfseries Вариант:} 22

$10^x -5x-2=0$
% \pagebreak

\subsection{Результаты работы}
\begin{figure}[h!]
\centering
\includegraphics[width=.9\textwidth]{lab2_1}
\caption{Вывод программы в консоли}
\end{figure}

% \vfill

% \begin{figure}[h!]
% \centering
% \includegraphics[width=.9\textwidth]{lab5_taylor}
% \caption{Решение с аппроксимацией граничных условий со вторым порядком}
% \end{figure}
\pagebreak

\subsection{Исходный код}
% \lstinputlisting[language=C++]{matrix.cpp}
% \begin{lstlisting}
\lstinputlisting{include/lab2_1.cpp}
% \end{lstlisting}
% \lstinputlisting{matrix.cpp}
% {../../include/matrix.cpp}
% \pagebreak
% \lstinputlisting[title=\texttt{parabolic\_pde.hpp}]{../../include/partial_differential/parabolic_pde.hpp}
% \pagebreak
% 
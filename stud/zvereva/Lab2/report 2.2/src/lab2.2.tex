\section* {2.2  Методы простой итерации и Ньютона}

\subsection{Постановка задачи}
Реализовать методы простой итерации и Ньютона решения систем нелинейных уравнений в виде программного кода, задавая в качестве входных данных точность вычислений. С использованием разработанного программного обеспечения решить систему нелинейных уравнений (при наличии нескольких решений найти то из них, в котором значения неизвестных являются положительными); начальное приближение определить графически. Проанализировать зависимость погрешности вычислений от количества итераций. 

{\bfseries Вариант:} 10

\begin{cases}
& x_1 - cosx_2 = 1 \\
& x_2 - sinx_1 = 1 \\
\end{cases}
%\pagebreak

\subsection{Результаты работы}
\begin{figure}[h!]
\centering
\includegraphics[width=16cm, height=5cm]{img}
\caption{Вывод программы в консоли}
\end{figure}
\pagebreak
% \vfill

% \begin{figure}[h!]
% \centering
% \includegraphics[width=.9\textwidth]{lab5_taylor}
% \caption{Решение с аппроксимацией граничных условий со вторым порядком}
% \end{figure}

\subsection{Исходный код}
% \lstinputlisting[language=C++]{matrix.cpp}
% \begin{lstlisting}
\lstinputlisting{include/L2.2.cpp}
% \end{lstlisting}
% \lstinputlisting{matrix.cpp}
% {../../include/matrix.cpp}
% \pagebreak
% \lstinputlisting[title=\texttt{parabolic\_pde.hpp}]{../../include/partial_differential/parabolic_pde.hpp}
% \pagebreak
% 